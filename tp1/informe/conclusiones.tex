%1. De haber diferentes tamaños de redes, ¿Aprecia alguna diferencia desde el punto de vista de las fuentes
%de información analizadas?
	En S1 pudimos observar diferencias en la cantidad de broadcast así
    como en la cantidad de trafico ARP, entre las diferentes muestras
    obtenidas. Por ejemplo en redes mas grandes tanto unas como otras
    disminuyeron. Esto nos hizo pensar que, a medida que la red tiene
    un tamaño más grande y disminuye el trafico de control, se aprecia
    un incremento en la presencia de goodput.

    
    Algo que notamos fue la presencia del protocolo LLC en las redes
    más grandes, es decir en la de la biblioteca y la del pabellón 1
    de la facultad.

    También notamos en las redes más grandes la presencia de nodos que
    nos pareció que pertenecían a distintas subredes, dado que en cada
    caso había dos conjuntos de números que compartían los bits más
    significativos variando en los menos significativos. Además, al
    realizar las gráficas de los grafos subyacentes a las mismas,
    pudo observarse que parece haber nodos especialmente dedicados a
    conectar las dos subredes.

    En S2 se pudo observar diferencias de porcentajes en el trafico de
    broadcast. Por ejemplo, en la red casa eric se tuvo un mayor
    porcentaje de broadcast comparándola con las otras 2, esta supera
    por el doble el porcentaje de broadcast.Esto hace llegar a la
    conclusión de que a medida que crece la cantidad de usuarios en la
    red, el trafico broadcast decrece de forma proporcionalmente baja.


%2. ¿Ha notado alguna diferencia durante la captura de datos entre el acceso a la red mediante WiFi y el
%acceso mediante cable? ¿A qué se lo atribuye?
Debido a ciertos problemas no se pudo hacer un análisis concluyente entre la diferencia entre cableado y wifi.

%3. ¿Considera que las muestras obtenidas analizadas son representativas del comportamiento general
%de la red?
Consideramos que es probable que las muestras obtenidas sean
representativas del comportamiento general de una red, si bien
consideramos que ciertos paquetes de control, necesarios para el buen
funcionamiento, pueden no haber sido capturados, por haber ocurrido en
una etapa anterior al inicio de la captura de los paquetes.


%4. El modelo de fuente de memoria nula utilizado supone que las probabilidades de los símbolos son
%independientes. ¿Es esto verdadero para ambas fuentes? ¿Por qué? ¿Qué consecuencia tiene esto en
%los experimentos realizados?

Podemos decir que el modelo de memoria nula no es del todo adecuado a
la situación modelada. En particular, muchos de los paquetes enviados
son repuestas a paquetes recibidos, es decir, a paquetes que fueron
enviados con anterioridad. Por otra parte, muchas veces los paquetes
son enviados en ráfagas, lo que hace que dos paquetes pertenecientes a
una de ellas no pueden considerarse como independientes.

Así, en el caso de S1, por ejemplo si ocurre una ráfaga de paquetes IP,
esto implica que la probabilidad de ocurrencia no sea independiente entre
ellos, ya que vendrán varios uno cercano al otro.

En S2, por ejemplo, un paquete is-at es respuesta a un paquete
who-has. Como nuestro modelo considera sólo los números de IP, este
hecho se ve reflejado en el hecho de que un número IP que aparece como
parte de una paquete is-at, es respuesta (y por ende no independiente)
a un paquete who.has.

%5. ¿Qué otra herramienta matemática distinta a la teoría de la información podría utilizar para detectar
%elementos distinguidos en una muestra de datos?

Otra herramienta matemática que puede ser de utilidad para analizar
una red y en particular para identificar nodos distinguidos dentro de
la misma es la teoría de grafos. En nuestro TP pudimos encontrar que
en ocasiones los nodos distinguidos son aquellos que mayor grado tienen
en el grado subyacente y también que conectan de tal modo el grafo que
si se los saca el grafo deja de ser conexo.
