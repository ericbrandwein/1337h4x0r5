% La presentación de los resultados debe efectuarse para cada red mediante, al menos, los gráficos suge-
% ridos a continuación:
% 1. Para S1: Mostrar la cantidad de infomación de cada símbolo comparando con la entropía de la fuente
% y la entropía máxima. Mostrar el porcentaje de tráfico broadcast sobre el tráfico total. Mostrar el
% porcentaje de aparición de cada protocolo encontrado.
% 2. Para S2: Mostrar la cantidad de información de cada símbolo comparando con la entropía de la fuente
% y la entropía máxima. Dados los paquetes ARP, mostrar mediante un grafo, la red de mensajes ARP
% subyacente (de ser necesario, agrupar adecuadamente varios nodos en uno para mejorar la visualización).

% A su vez los resultados por experimento deben responder para cada red, las preguntas descriptas a
% continuación (no hace falta transcribirlas en el informe y se valorará significativamente el planteo de nuevas
% preguntas):
% 1. Para S1: ¿Considera significativa la cantidad de tráfico broadcast sobre el tráfico total? 

Para cada muestra, generamos un gráfico mostrando la información de cada símbolo,
la entropía de la muestra, y la entropía máxima.

\subsection{Casa de Eric}
El porcentaje de paquetes de broadcast sobre los totales en los datos tomados en la 
casa de Eric es del 31,8\%. Esto quiere decir que estos paquetes no son los de 
mayor aparición. A continuación se presenta un gráfico que ayudará a analizar mejor
la muestra:

\includegraphics{s1/casa-eric-nuevo.png}


¿Cuál es la función de cada uno de los protocolos encontrados? 
¿Cuáles son de control y cuáles transportan datos de usuario? 
\textbf{ESTO NO LO SÉ, RESPONDANLO USTEDES.}
% ¿Se han encontrado símbolos distinguidos? ¿Les otorga algún significado?

% TODO

% ¿La entropía de la fuente es máxima? ¿Bajo qué condiciones la entropía sería máxima? 
Se puede ver que la entropía de la fuente es menor a la entropía máxima. Esto
se debe a que no es uniforme la información de todos los símbolos: el símbolo
IP:unicast contiene mucha más información que el IP:broadcast. En términos más
palpables, esto quiere decir que es más predecible la red. Si uno tuviese que
adivinar cuál será el próximo paquete recibido, decir que será de tipo
IP:broadcast tendría muchas chances de ser correcto.


% ¿Ha encontrado protocolos no esperados? ¿Puede describirlos?

% ... hablando de la red de los labos...
Encontramos que algunos paquetes tenían protocolo LLC. Investigando, descubrimos
que es un protocolo de capa dos que permite multiplexar muchas fuentes de
información en una sola. En el caso particular de los laboratorios no
estamos seguros del uso que se les está dando.




% 2. Para S2: ¿La entropía de la fuente es máxima? ¿Qué sugiere esto acerca de la red? ¿Bajo qué con-
% diciones la entropía sería máxima? ¿Se pueden distinguir nodos? ¿Se les puede adjudicar alguna
% función específica? ¿Hay evidencia parcial que sugiera que algún nodo funciona de forma anómala
% y/o no esperada? ¿Existe una correspondencia entre lo que se conoce de la red y los nodos distinguidos
% detectados por la herramienta? ¿Ha encontrado paquetes ARP no esperados? ¿Cuál es su función?

\subsection{Biblioteca Lavalleja}
% 1. Para S1: Mostrar la cantidad de infomación de cada símbolo comparando con la entropía de la fuente
% y la entropía máxima. Mostrar el porcentaje de tráfico broadcast sobre el tráfico total. Mostrar el
% porcentaje de aparición de cada protocolo encontrado.

% 1. Para S1: ¿Considera significativa la cantidad de tráfico broadcast sobre el tráfico total? 
El porcentaje de tráfico de broadcast en esta red es del 14,5\%. El tráfico unicast supera ampliamente
al broadcast en esta red en particular. Esto podría ocasionarse por el hecho de que los
usuarios de la biblioteca utilizan la red en conexiones largas, y por lo tanto los paquetes
de control de red son opacados por los de comunicación de datos. A continuación se muestra un
gráfico que nos ayudará a analizar mejor la muestra.

% GRAFICULO


Se puede ver que el símbolo LLC:broadcast es el que aporta mayor información.
Suponemos que son paquetes internos de la biblioteca, y que entonces son
superados en número por los paquetes enviados por los usuarios de la biblioteca.
También se puede ver que los IP:broadcast son los que menos información brindan.
Esto quiere decir que son los más enviados. 
\textbf{JUSTIFICAR POR QUÉ HAY MENOS BROADCAST QUE UNICAST PERO EN EL GRÁFICO
PARECE LO CONTRARIO}.


¿Cuál es la función de cada uno de los protocolos encontrados? 
¿Cuáles son de control y cuáles transportan datos de usuario? 
\textbf{ESTO NO LO SÉ, RESPONDANLO USTEDES.}
% ¿Se han encontrado símbolos distinguidos? ¿Les otorga algún significado?

% TODO

% ¿La entropía de la fuente es máxima? ¿Bajo qué condiciones la entropía sería máxima?
Podemos ver que la entropía en este caso tampoco es máxima, ya que hay mucha 
diferencia entre el símbolo de mayor información y el de menor información.
% ¿Ha encontrado protocolos no esperados? ¿Puede describirlos?
