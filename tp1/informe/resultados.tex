% La presentación de los resultados debe efectuarse para cada red mediante, al menos, los gráficos suge-
% ridos a continuación:
% 1. Para S1: Mostrar la cantidad de infomación de cada símbolo comparando con la entropía de la fuente
% y la entropía máxima. Mostrar el porcentaje de tráfico broadcast sobre el tráfico total. Mostrar el
% porcentaje de aparición de cada protocolo encontrado.
% 2. Para S2: Mostrar la cantidad de información de cada símbolo comparando con la entropía de la fuente
% y la entropía máxima. Dados los paquetes ARP, mostrar mediante un grafo, la red de mensajes ARP
% subyacente (de ser necesario, agrupar adecuadamente varios nodos en uno para mejorar la visualización).

% A su vez los resultados por experimento deben responder para cada red, las preguntas descriptas a
% continuación (no hace falta transcribirlas en el informe y se valorará significativamente el planteo de nuevas
% preguntas):
% 1. Para S1: ¿Considera significativa la cantidad de tráfico broadcast sobre el tráfico total? 

Para cada muestra, generamos un gráfico mostrando la información de cada símbolo,
la entropía de la muestra, y la entropía máxima.

\subsection{Casa de Eric}
\subsubsection{S1}
El porcentaje de paquetes de broadcast sobre los totales en los datos tomados en la 
casa de Eric es del 31,8\%. Esto quiere decir que estos paquetes no son los de 
mayor aparición. A continuación se presenta un gráfico que ayudará a analizar mejor
la muestra:

\includegraphics{s1/casa-eric-nuevo.png}


% ¿Se han encontrado símbolos distinguidos? ¿Les otorga algún significado?
% ¿La entropía de la fuente es máxima? ¿Bajo qué condiciones la entropía sería máxima? 
Se puede ver que la entropía de la fuente es menor a la entropía máxima. Esto
se debe a que no es uniforme la información de todos los símbolos: el símbolo
IP:broadcast contiene mucha más información que el IP:unicast. En términos más
palpables, esto quiere decir que es más predecible la red. Si uno tuviese que
adivinar cuál será el próximo paquete recibido, decir que será de tipo
IP:unicast tendría muchas chances de ser correcto.

Además, el símbolo IP:unicast es el único que posee información menor a la
entropía, haciéndolo mucho más común que los otros símbolos. De 45.143 paquetes
muestreados, 18.207 tuvieron este símbolo, haciendo que el 40,3\% de los paquetes
sean de este símbolo. Este es, definitivamente, un símbolo distinguido.

Como los paquetes de IP:unicast transportan en su mayoría datos de usuario,
que este símbolo sea distinguido indica que la red tiene buen goodput.

% ¿Ha encontrado protocolos no esperados? ¿Puede describirlos?


% 2. Para S2: ¿La entropía de la fuente es máxima? ¿Qué sugiere esto acerca de la red? ¿Bajo qué con-
% diciones la entropía sería máxima? ¿Se pueden distinguir nodos? ¿Se les puede adjudicar alguna
% función específica? ¿Hay evidencia parcial que sugiera que algún nodo funciona de forma anómala
% y/o no esperada? ¿Existe una correspondencia entre lo que se conoce de la red y los nodos distinguidos
% detectados por la herramienta? ¿Ha encontrado paquetes ARP no esperados? ¿Cuál es su función?

\subsection{Biblioteca Casa de la Lectura}
\subsubsection{S1}
% 1. Para S1: Mostrar la cantidad de infomación de cada símbolo comparando con la entropía de la fuente
% y la entropía máxima. Mostrar el porcentaje de tráfico broadcast sobre el tráfico total. Mostrar el
% porcentaje de aparición de cada protocolo encontrado.

% 1. Para S1: ¿Considera significativa la cantidad de tráfico broadcast sobre el tráfico total? 
El porcentaje de tráfico de broadcast en esta red es del 14,5\%. El tráfico unicast supera ampliamente
al broadcast en esta red en particular. Esto podría ocasionarse por el hecho de que los
usuarios de la biblioteca utilizan la red en conexiones largas, y por lo tanto los paquetes
de control de red son opacados por los de comunicación de datos. A continuación se muestra un
gráfico que nos ayudará a analizar mejor la muestra.

% GRAFICULO

%¿Cuál es la función de cada uno de los protocolos encontrados? 
%¿Cuáles son de control y cuáles transportan datos de usuario? 

Se puede ver que el símbolo LLC:unicast es el que aporta mayor información.
Suponemos que son paquetes internos de la biblioteca, y que entonces son
superados en número por los paquetes enviados por los usuarios de la biblioteca.
También se puede ver que los IP:unicast son los que menos información brindan,
seguidos por los IPv6:unicast. Esto quiere decir que son los más enviados.
Estos son los paquetes que en su mayoría contienen datos de usuario,
a diferencia de los de ARP, que sirven para relacionar las direcciones IP con
las MAC de sus dueños; y los de IP:broadcast. Estos se utilizan para control de red.


% ¿Se han encontrado símbolos distinguidos? ¿Les otorga algún significado?

De nuevo, el único símbolo distinguido es el IP:unicast, que es el único cuya
información es menor a la entropía.

% ¿La entropía de la fuente es máxima? ¿Bajo qué condiciones la entropía sería máxima?
Podemos ver que la entropía en este caso tampoco es máxima, ya que hay mucha 
diferencia entre el símbolo de mayor información y el de menor información.

% ¿Ha encontrado protocolos no esperados? ¿Puede describirlos?
Podemos ver que algunos paquetes tienen protocolo LLC. Investigando, descubrimos
que es un protocolo de capa dos que permite multiplexar muchas fuentes de
información en una sola. En el caso particular de la biblioteca no
estamos seguros del uso que se les está dando.


\subsection{Red Pabellón 1}
\subsubsection{S1}
El porcentaje de tráfico de broadcast en esta red fue del 15,3\%. De nuevo, el
tráfico unicast lo supera ampliamente. Suponemos que ocurre lo mismo que en 
la muestra anterior: al tener más usuarios de la red enviando datos en 
sesiones largas, la cantidad de paquetes de control de red necesarios 
decrementa, que son en su mayoría de broadcast.
Esto podría derivar en un aumento del goodput, queriendo decir que cuantos 
más usuarios, mayor goodput.

A continuación se muestra el gráfico de barras correspondiente a esta red.

% GRAFICULIS

%¿Cuál es la función de cada uno de los protocolos encontrados? 
%¿Cuáles son de control y cuáles transportan datos de usuario? 
Acá vemos más marcada la diferencia entre la información otorgada por el
símbolo IP:unicast y los demás. Además de indicar un mayor goodput, 
hace que la entropía baje en relación a la entropía máxima, ya que la
red es más predecible bajo este modelo. También está representado el
símbolo IPv6:unicast, lo que quiere decir que hay algunas comunicaciones
que se realizan con este protocolo más moderno.

% ¿Se han encontrado símbolos distinguidos? ¿Les otorga algún significado?

Una vez más podemos ver que el único símbolo distinguido es el IP:unicast.
% ¿La entropía de la fuente es máxima? ¿Bajo qué condiciones la entropía sería máxima?


% ¿Ha encontrado protocolos no esperados? ¿Puede describirlos?
En este caso se están mandando paquetes LLC:broadcast, a diferencia del caso
de la biblioteca, en el que se mandaban de LLC:unicast. Podemos inferir solamente
que el protocolo LLC es una forma genérica de enviar paquetes que cada administrador
de red puede adaptar a sus necesidades.