%Entre los métodos y condiciones de cada
%experimento se debe detallar la descripción de la red -tipo, tamaño, modo de acceso, etc.-, las características
%de la muestra -tamaño, horario, día de la semana, etc.- y la justificación de la elección del modelo de la fuente
%S2.
%La presentación de los resultados debe efectuarse para cada red mediante, a

% Explicar las fuentes de información S1 y S2.
Para evaluar las características de las redes utilizamos dos diferentes modelos
de fuentes de información de memoria nula. La primera, que llamamos S1, contiene
los símbolos compuestos por la combinación del tipo de destino de la trama
(unicast o broadcast) y el protocolo de la capa inmediata superior encapsulado
en la misma. La segunda, llamada S2, \textbf{COMPLETAR}.

\subsection{Casa de Eric}
El primer experimento fue realizado en la casa de uno de los integrantes del grupo, por
medio de una conexión inalámbrica. Es una red en la que hay una combinación de dispositivos
móviles, laptops, computadoras de escritorio, un router, y tres repetidores, 
que suman entre 8 y 16 dispositivos conectados al mismo tiempo, dependiendo del momento.
Las muestras fueron tomadas en fin de semana a la tarde. 
Se capturaron 89.675 paquetes
para la fuente de información S1, y 14.185 paquetes para S2.

Podemos considerar a esta como una red de tamaño pequeño.

\subsection{Biblioteca Casa de la Lectura}
Esta muestra fue tomada en la biblioteca Casa de la Lectura, ubicada en
la calle Lavalleja al 924, Villa Crespo, con una conexión WiFi desde una laptop.
La fecha fue el 30 de abril entre las
15:30 y 16:30. En el momento se encontraban alrededor de 20 personas en el lugar,
algunos utilizando las computadoras de escritorio conectadas por cable
pertenecientes al lugar, y otras con sus propias laptops y celulares. 
Se capturaron 11000 paquetes para la fuente de información S1, 
y 12277 paquetes para la fuente S2.

Consideramos a esta como una red de mediano tamaño.

\subsection{Red Pabellón 1}
Se generaron estos datos registrando los paquetes enviados por medio de 
la red WiFi del pabellón 1 de la FCEN, mediante una laptop. Esta es una 
red grande, ya que hay docenas de dispositivos activos conectados al mismo
tiempo. Se tomó la muestra el Lunes 29 de Abril, alrededor de las 17 horas.
Se capturaron 24628 paquetes para la fuente de información S1, y 10909
paquetes para la fuente S2.
