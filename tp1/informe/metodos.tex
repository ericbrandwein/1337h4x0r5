%Entre los métodos y condiciones de cada
%experimento se debe detallar la descripción de la red -tipo, tamaño, modo de acceso, etc.-, las características
%de la muestra -tamaño, horario, día de la semana, etc.- y la justificación de la elección del modelo de la fuente
%S2.
%La presentación de los resultados debe efectuarse para cada red mediante, a

% Explicar las fuentes de información S1 y S2.
Para evaluar las características de las redes utilizamos dos diferentes modelos
de fuentes de información de memoria nula. La primera, que llamamos S1, contiene
los símbolos compuestos por la combinación del tipo de destino de la trama
(unicast o broadcast) y el protocolo de la capa inmediata superior encapsulado
en la misma. La segunda, llamada S2, \textbf{COMPLETAR}.

\subsection{Casa de Eric}
El primer experimento fue realizado en la casa de uno de los integrantes del grupo, por
medio de una conexión inalámbrica. Es una red en la que hay una combinación de dispositivos
móviles, laptops, computadoras de escritorio, un router, y tres repetidores, 
que suman entre 8 y 16 dispositivos conectados al mismo tiempo, dependiendo del momento.
Las muestras fueron tomadas en fin de semana a la tarde. Se capturaron 89.675 paquetes
para la fuente de información S1, y 14.185 paquetes para S2.