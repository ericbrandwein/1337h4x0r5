\subsection{Universidad de Oxford}
Esta muestra se realizó desde la casa de Franco, que tiene a Fibertel 
como proveedor de Internet. Se hizo a la página web de la Universidad de Oxford,
ox.ac.uk, de IP 129.67.242.154. 

A continuación se muestra la tabla que representa los datos extraídos por la
misma.

\begin{center}
\begin{tabular}{lcccc}
Destino & TTL & RTT & Zona Horaria & País\\
\hline
192.168.0.1 & 1 & 0.04538869 & NA & NA\\
200.89.165.86 & 5 & 0.05430764 & NA & AR\\
200.89.165.149 & 6 & 0.05146670 & NA & AR\\
200.89.165.85 & 6 & 0.05371330 & NA & AR\\
200.89.165.1 & 7 & 0.05211499 & NA & AR\\
200.89.165.130 & 7 & 0.05384665 & NA & AR\\
200.89.165.222 & 8 & 0.05457461 & NA & AR\\
200.89.165.250 & 8 & 0.05595906 & NA & AR\\
190.216.88.33 & 9 & 0.05468848 & Buenos Aires & AR\\
67.17.94.249 & 10 & 0.17580819 & NA & US\\
67.17.99.233 & 10 & 0.18295765 & NA & US\\
4.69.207.29 & 12 & 0.17595530 & NA & US\\
213.248.84.80 & 13 & 0.17912588 & Vaduz & NA\\
62.115.119.230 & 14 & 0.29029322 & Vaduz & NA\\
62.115.120.176 & 14 & 0.30006517 & Vaduz & NA\\
62.115.136.200 & 15 & 0.30010971 & Vaduz & NA\\
62.115.141.245 & 15 & 0.29688158 & Vaduz & NA\\
62.115.112.245 & 16 & 0.29005424 & Vaduz & NA\\
62.115.113.21 & 16 & 0.29445747 & Vaduz & NA\\
213.155.132.195 & 17 & 0.30028347 & Vaduz & NA\\
213.155.132.197 & 17 & 0.29843348 & Vaduz & NA\\
62.115.148.161 & 18 & 0.29240237 & Vaduz & NA\\
146.97.35.193 & 19 & 0.28767490 & London & GB\\
146.97.33.62 & 20 & 0.30540899 & London & GB\\
146.97.33.5 & 21 & 0.28970429 & London & GB\\
146.97.37.194 & 22 & 0.29844862 & London & GB\\
193.63.108.94 & 23 & 0.30389028 & London & GB\\
193.63.108.98 & 24 & 0.29854251 & London & GB\\
193.63.109.90 & 25 & 0.30345439 & London & GB\\
192.76.32.62 & 27 & 0.30002125 & London & GB\\
129.67.242.154 & 28 & 0.29852787 & London & GB\\
129.67.242.155 & 28 & 0.29153364 & London & GB\\
\end{tabular}
\end{center}


Lo primero que podemos ver es que hay muchos saltos con TTL repetidos. Podemos
asumir que esto es debido al \emph{load balancing}; la técnica de repartir
paquetes con el mismo destino por distintas rutas, para no congestionar una
ruta en particular. Lo malo de esto es que no tenemos un RTT único definido,
sino que tenemos uno por cada IP con igual TTL, y por lo tanto no podemos
definir la diferencia de RTT con precisión entre dos saltos.
Para solucionar esto, tomamos el RTT del salto como el promedio de los RTTs
de los distintos routers con el mismo TTL.

Por otra parte, podemos ver que paquetes con algunos TTL no obtuvieron
respuesta. Por ejemplo, desde el TTL 2 hasta el 4 no se obtuvo información. 
Tampoco del 11 y del 26. Esto podría llegar a afectar a los cálculos de outliers.

A continuación veremos la tabla con los RTTs de respuestas con TTLs repetidos
promediados.

% Created 2019-06-20 jue 23:12
% Intended LaTeX compiler: pdflatex
\documentclass[11pt]{article}
\usepackage[utf8]{inputenc}
\usepackage[T1]{fontenc}
\usepackage{graphicx}
\usepackage{grffile}
\usepackage{longtable}
\usepackage{wrapfig}
\usepackage{rotating}
\usepackage[normalem]{ulem}
\usepackage{amsmath}
\usepackage{textcomp}
\usepackage{amssymb}
\usepackage{capt-of}
\usepackage{hyperref}
\author{nsm}
\date{\today}
\title{}
\hypersetup{
 pdfauthor={nsm},
 pdftitle={},
 pdfkeywords={},
 pdfsubject={},
 pdfcreator={Emacs 26.1 (Org mode 9.2.3)}, 
 pdflang={English}}
\begin{document}

\tableofcontents

\begin{center}
\begin{tabular}{rrrlll}
ttl & mean.rtt & deltas & outlier & Timezone & Pais\\
\hline
1 & 0.04538869 & 4.538869e-02 & TRUE & <NA> & <NA>\\
5 & 0.05430764 & 8.918954e-03 & FALSE & None & AR\\
6 & 0.05146670 & 0.000000e+00 & FALSE & None & AR\\
7 & 0.05211499 & 0.000000e+00 & FALSE & None & AR\\
8 & 0.05457461 & 2.669692e-04 & FALSE & None & AR\\
9 & 0.05468848 & 0.000000e+00 & FALSE & Buenos\textsubscript{Aires} & AR\\
10 & 0.17580819 & 1.198491e-01 & TRUE & None & US\\
12 & 0.17595530 & 0.000000e+00 & FALSE & None & US\\
13 & 0.17912588 & 0.000000e+00 & FALSE & Europe/Vaduz & None\\
14 & 0.29029322 & 1.073356e-01 & TRUE & Europe/Vaduz & None\\
15 & 0.30010971 & 4.454475e-05 & FALSE & Europe/Vaduz & None\\
16 & 0.29005424 & 0.000000e+00 & FALSE & Europe/Vaduz & None\\
17 & 0.30028347 & 1.737545e-04 & FALSE & Europe/Vaduz & None\\
18 & 0.29240237 & 0.000000e+00 & FALSE & Europe/Vaduz & None\\
19 & 0.28767490 & 0.000000e+00 & FALSE & Europe/London & GB\\
20 & 0.30540899 & 5.125526e-03 & FALSE & Europe/London & GB\\
21 & 0.28970429 & 0.000000e+00 & FALSE & Europe/London & GB\\
22 & 0.29844862 & 0.000000e+00 & FALSE & Europe/London & GB\\
23 & 0.30389028 & 0.000000e+00 & FALSE & Europe/London & GB\\
24 & 0.29854251 & 0.000000e+00 & FALSE & Europe/London & GB\\
25 & 0.30345439 & 0.000000e+00 & FALSE & Europe/London & GB\\
27 & 0.30002125 & 0.000000e+00 & FALSE & Europe/London & GB\\
28 & 0.29852787 & 0.000000e+00 & FALSE & Europe/London & GB\\
\end{tabular}
\end{center}
\end{document}


El mapa de conexiones correspondiente es:

\begin{center}
    \includegraphics[scale=0.4]{fran-oxford.png}
\end{center}
    

Podemos ver que hay 4 outliers. El primero es el del primer hop, que corresponde
al hop de la computadora al router WiFi. Suponemos que esto se podría deber a
un conjunto de cosas variadas:
\begin{enumerate}
\item El hecho de que el traceroute se hizo con la URL de la universidad en vez de 
la IP. Esto dispara una resolución DNS, lo cual tarda.
\item El hecho de que la placa de red de la computadora sea más lenta
que la de los routers en general.
\item El hecho de que estamos utilzando WiFi, lo cual es más lento que una conexión
cableada.
\end{enumerate}

El segundo outlier se encuentra en el hop de TTL 10, que corresponde al salto
entre Argentina y Estados Unidos. Esto parece tener sentido, y parece indicar
correctamente que entre la Argentina y Estados Unidos existe una conexión de
larga distancia, aunque no se pueda considerar intercontinental.

El tercero y el cuarto, en cambio, no parecen reflejar correctamente saltos
intercontinentales. Si analizamos mejor los datos, nos damos cuenta que ningún
TTL desde el 14 supera a un RTT de alrededor de 0,3ms. Podría estar ocurriendo,
como explicamos previamente, que hubiese un MPLS que esté compuesto de algunos
o todos de los routers en esa zona. También podría ocurrir que simplemente
la cantidad de paquetes enviados no haya sido suficiente para representar los
RTT verdaderos, y por lo tanto aparecen con un desvío mayor al real. Otra causa
de la predicción errónea del tercer outlier es la falsa respuesta del servicio
de ubicación que utilizamos. Es posible que la primer IP de Europa sea en
realidad de Estados Unidos, y entonces el cálculo de outliers en realidad fue
certero.

\subsection{Universidad de Tel Aviv}
Esta muestra se realizó desde la casa de Nicolás, con destino a la página web
de la Universidad de Tel Aviv, de URL tau.ac.il e IP 132.66.11.168.
A continuación se muestran los resultados obtenidos.

% Created 2019-06-20 jue 22:07
% Intended LaTeX compiler: pdflatex
\documentclass[11pt]{article}
\usepackage[utf8]{inputenc}
\usepackage[T1]{fontenc}
\usepackage{graphicx}
\usepackage{grffile}
\usepackage{longtable}
\usepackage{wrapfig}
\usepackage{rotating}
\usepackage[normalem]{ulem}
\usepackage{amsmath}
\usepackage{textcomp}
\usepackage{amssymb}
\usepackage{capt-of}
\usepackage{hyperref}
\author{nsm}
\date{\today}
\title{}
\hypersetup{
 pdfauthor={nsm},
 pdftitle={},
 pdfkeywords={},
 pdfsubject={},
 pdfcreator={Emacs 26.1 (Org mode 9.2.3)}, 
 pdflang={English}}
\begin{document}

\tableofcontents

\begin{center}
\begin{tabular}{rrrrlll}
dst & ttl & mean.rtt & deltas & outlier & Timezone & Pais\\
\hline
192.168.0.1 & 1 & 0.05399479 & 0.053994794 & FALSE & <NA> & <NA>\\
10.0.0.2 & 2 & 0.05373263 & 0.000000000 & FALSE & <NA> & <NA>\\
181.88.172.60 & 3 & 0.06836800 & 0.014635369 & FALSE & Cordoba & AR\\
181.89.5.92 & 4 & 0.06620540 & 0.000000000 & FALSE & None & AR\\
181.88.111.26 & 5 & 0.07400775 & 0.007802352 & FALSE & None & AR\\
185.70.203.48 & 6 & 0.07128960 & 0.000000000 & FALSE & Europe/Rome & IT\\
89.221.41.161 & 7 & 0.20464959 & 0.133359991 & TRUE & Europe/Rome & IT\\
154.54.9.17 & 8 & 0.20224162 & 0.000000000 & FALSE & None & US\\
154.54.47.29 & 9 & 0.20027758 & 0.000000000 & FALSE & None & US\\
154.54.28.49 & 10 & 0.21373808 & 0.013460502 & FALSE & None & US\\
154.54.24.221 & 11 & 0.22232039 & 0.008582309 & FALSE & None & US\\
154.54.40.109 & 12 & 0.22999876 & 0.007678367 & FALSE & None & US\\
154.54.42.86 & 13 & 0.28045648 & 0.050457723 & FALSE & None & US\\
130.117.51.42 & 14 & 0.28898980 & 0.008533321 & FALSE & Europe/Vaduz & None\\
130.117.0.142 & 15 & 0.32101583 & 0.032026023 & FALSE & Europe/Vaduz & None\\
130.117.1.117 & 16 & 0.35113116 & 0.030115329 & FALSE & Europe/Vaduz & None\\
149.29.9.10 & 17 & 0.31966097 & 0.000000000 & FALSE & None & US\\
62.40.98.129 & 19 & 0.33959182 & 0.019930854 & FALSE & Europe/London & GB\\
83.97.88.94 & 20 & 0.39568052 & 0.056088693 & FALSE & Europe/London & GB\\
28.139.191.69 & 21 & 0.38993513 & 0.000000000 & FALSE & Asia/Jerusalem & IL\\
132.66.4.254 & 22 & 0.39242335 & 0.002488217 & FALSE & Asia/Jerusalem & IL\\
132.66.11.168 & 23 & 0.38219708 & 0.000000000 & FALSE & Asia/Jerusalem & IL\\
\end{tabular}
\end{center}
\end{document}


El mapa de conexiones correspondiente es:

\begin{center}
    \includegraphics[scale=0.4]{nico-icmp-tel-aviv.png}
\end{center}

Claramente se puede ver que el servicio de ubicación utilizado no es el más
confiable, ya que no existe una conexión directa de Argentina a Italia, como
parecen indicar los paquetes de TTL 5 y 6.
La verdad es que muchos servicios fueron probados para descubrir
cuál es el mejor, pero ninguno pareció resultar confiable.
Podemos asumir que esas dos IPs en cambio se encuentran en Estados Unidos,
lo que sería lógico, tomando en cuenta que las siguientes son también de 
Estados Unidos. Esta suposición, de todas formas, no se condice con el único
outlier que encontramos, que es el segundo router de Roma. Por lo tanto,
otra posibilidad es que el servicio ubica una IP en Estados Unidos cuando no
está seguro de la ubicación, y que la primera IP del TTL 6 en realidad es 
de Estados Unidos. Así, el cálculo de outliers estaría en lo correcto. 
Tendríamos ahora el problema de que la diferencia del TTL 6 es 0, lo cual no
estaría indicando una conexión de larga distancia como ser la de Argentina a Estados
Unidos.

El mapa también muestra que se va y vuelve de Europa a Estados Unidos continuamente.
Esto también sospechamos que es culpa del servicio de ubicación. Es difícil,
por lo tanto, extraer conclusiones de esta muestra.
