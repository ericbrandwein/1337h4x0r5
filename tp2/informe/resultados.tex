\subsection{Universidad de Oxford}
Esta muestra se realizó desde la casa de Franco, que tiene a Fibertel 
como proveedor de Internet. Se hizo a la página web de la Universidad de Oxford,
ox.ac.uk, de IP 129.67.242.154. 

A continuación se muestra la tabla que representa los datos extraídos por la
misma.

\begin{center}
\begin{tabular}{lcccc}
Destino & TTL & RTT & Zona Horaria & País\\
\hline
192.168.0.1 & 1 & 0.04538869 & NA & NA\\
200.89.165.86 & 5 & 0.05430764 & NA & AR\\
200.89.165.149 & 6 & 0.05146670 & NA & AR\\
200.89.165.85 & 6 & 0.05371330 & NA & AR\\
200.89.165.1 & 7 & 0.05211499 & NA & AR\\
200.89.165.130 & 7 & 0.05384665 & NA & AR\\
200.89.165.222 & 8 & 0.05457461 & NA & AR\\
200.89.165.250 & 8 & 0.05595906 & NA & AR\\
190.216.88.33 & 9 & 0.05468848 & Buenos Aires & AR\\
67.17.94.249 & 10 & 0.17580819 & NA & US\\
67.17.99.233 & 10 & 0.18295765 & NA & US\\
4.69.207.29 & 12 & 0.17595530 & NA & US\\
213.248.84.80 & 13 & 0.17912588 & Vaduz & NA\\
62.115.119.230 & 14 & 0.29029322 & Vaduz & NA\\
62.115.120.176 & 14 & 0.30006517 & Vaduz & NA\\
62.115.136.200 & 15 & 0.30010971 & Vaduz & NA\\
62.115.141.245 & 15 & 0.29688158 & Vaduz & NA\\
62.115.112.245 & 16 & 0.29005424 & Vaduz & NA\\
62.115.113.21 & 16 & 0.29445747 & Vaduz & NA\\
213.155.132.195 & 17 & 0.30028347 & Vaduz & NA\\
213.155.132.197 & 17 & 0.29843348 & Vaduz & NA\\
62.115.148.161 & 18 & 0.29240237 & Vaduz & NA\\
146.97.35.193 & 19 & 0.28767490 & London & GB\\
146.97.33.62 & 20 & 0.30540899 & London & GB\\
146.97.33.5 & 21 & 0.28970429 & London & GB\\
146.97.37.194 & 22 & 0.29844862 & London & GB\\
193.63.108.94 & 23 & 0.30389028 & London & GB\\
193.63.108.98 & 24 & 0.29854251 & London & GB\\
193.63.109.90 & 25 & 0.30345439 & London & GB\\
192.76.32.62 & 27 & 0.30002125 & London & GB\\
129.67.242.154 & 28 & 0.29852787 & London & GB\\
129.67.242.155 & 28 & 0.29153364 & London & GB\\
\end{tabular}
\end{center}


Lo primero que podemos ver es que hay muchos saltos con TTL repetidos. Podemos
asumir que esto es debido al \emph{load balancing}; la técnica de repartir
paquetes con el mismo destino por distintas rutas, para no congestionar una
ruta en particular. Lo malo de esto es que no tenemos un RTT único definido,
sino que tenemos uno por cada IP con igual TTL, y por lo tanto no podemos
definir la diferencia de RTT con precisión entre dos saltos.
Para solucionar esto, tomamos el RTT del salto como el promedio de los RTTs
de los distintos routers con el mismo TTL.

Por otra parte, podemos ver que paquetes con algunos TTL no obtuvieron
respuesta. Por ejemplo, desde el TTL 2 hasta el 4 no se obtuvo información. 
Tampoco del 11 y del 26. Esto podría llegar a afectar a los cálculos de outliers.



\subsection{Universidad de Tel Aviv}
