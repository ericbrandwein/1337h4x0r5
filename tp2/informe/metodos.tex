\subsection{Herramienta de Traceroute en Python}
Para implementar la herramienta símil Traceroute utilizamos la
librería Scapy de Python, que permite, entre otras cosas,
enviar y recibir paquetes ICMP, TCP y UDP a una dirección IP,
o a una URL, la cual es convertida a dirección IP por medio de 
DNS. Llamamos a la herramienta desarrollada traceroute.py.

Este programa tiene como argumento la IP o URL destino de la ruta
a trazar, y envía paquetes incrementando de a uno el TTL. El TTL
se decrementa en uno por cada router por el que pasa el paquete, y cuando
llega a tener TTL igual a 0, el router que lo recibió envía un mensaje de
error al emisor del paquete. Así, podemos saber las direcciones IP de (casi)
todos los routers de un camino posible a la IP destino. Además, podemos saber
cuánto se tardó entre el envío de un paquete y la respuesta del router, que 
corresponde al RTT. Restando el RTT de un paquete del RTT del paquete con TTL
incrementado en uno podemos estimar el RTT entre un router y el siguiente
en la ruta.

Puede ocurrir que la IP destino no responda al paquete enviado, o que la IP
sea inalcanzable. Para salvarnos de estos casos, la herramienta incrementa
el TTL hasta 32, y si no logra alcanzar el destino hasta ese momento corta la
ejecución.

Para lidiar con variaciones en los RTT de las rutas y de los routers de cada hop
de cada paquete de TTL distinto,
se envían una cantidad configurable de paquetes por cada TTL, con valor por
defecto 32. Los valores de los RTT de estos paquetes serán luego procesados con
scripts en R para obtener un RTT entre routers aproximados.

El programa escribe la IP destino, el RTT y el TTL de cada paquete que envía
a un archivo CSV. Si el destino no es sabido debido a que el router que envió
el mensaje de \emph{Time Exceeded} no incluyó su IP, se imprime ``NA'' en su 
lugar. 

\subsection{Análisis de datos en R}
Para analizar los datos resultantes desarrollamos distintas herramientas en R.

Lo primero que se hace es filtrar las entradas de los CSVs según métodos 
explicados más adelante.
Luego, se calcula la media de los RTT de los
paquetes de cada TTL, para estimar el RTT de un paquete promedio al router
correspondiente. Con estos RTTs promedio calculamos la diferencia entre un 
router y el siguiente en la ruta, para luego poder identificar los enlaces
más lentos. Luego, para cada ruta encontramos los RTT de mayor magnitud con
el método de identificación de outliers propuesta por Cimbala. Además,
obtendremos las ubicaciones en el mapa de los routers por los que pasa 
el paquete con un servicio de Internet que la obtiene según una dirección IP.
Contrastaremos estas dos fuentes de información para ver si se condicen las
conexiones de Internet intercontinentales con los aumentos de RTT. 

Los paquetes para los que no hubo respuesta son obviados de este cálculo. 
Por lo tanto, es posible que ocurra que no tenemos información para alguno de 
los hops de una ruta. Lo que hacemos en este caso es hacer como que el hop no
existe, y calcular su RTT como igual al del router anterior. Esto podría
producir resultados no tan fieles a la realidad, porque se podrían estar
combinando los tiempos de tardanza de muchos routers diferentes en uno solo.

Algunos paquetes, además, parecen tardar menos en alcanzar algunos routers
en la ruta en comparación a routers anteriores, lo cual no tiene sentido
físico. Los RTTs que nos resultan negativos debido a estas particularidades
los ignoramos, y tomamos en cambio como si hubiese RTT igual a 0. Una
suposición que podemos hacer es que esos routers posteriores tienen menos carga
de paquetes o son más rápidos en responder paquetes de su buffer, y por lo
tanto su respuesta es más rápida que la de los routers anteriores.

Un último filtro aplicado es el de eliminar los primeros hops, que pertenecen
a routers de IP privada. Una de las razones es que el primer hop parece
tardar demasiado, probablemente por el hecho de que estamos utilizando una 
conexión WiFi para extraer los datos. Otra razón es la de que no nos parecen
relevantes para el análisis de la ruta en términos generales; no nos parece que
aporten información suficientemente útil.
