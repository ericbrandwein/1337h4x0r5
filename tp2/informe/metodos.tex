\subsection{Herramienta de traceroute en Python}
Para implementar la herramienta símil Traceroute utilizamos la
librería Scapy de Python, que permite, entre otras cosas,
enviar y recibir paquetes ICMP, TCP y UDP a una dirección IP,
o a una URL, la cual es convertida a dirección IP por medio de 
DNS. Llamamos a la herramienta desarrollada traceroute.py.

Este programa tiene como argumento la IP o URL destino de la ruta
a trazar, y envía paquetes incrementando de a uno el TTL. El TTL
se decrementa en uno por cada router por el que pasa el paquete, y cuando
llega a tener TTL igual a 0, el router que lo recibió envía un mensaje de
error al emisor del paquete. Así, podemos saber las direcciones IP de (casi)
todos los routers de un camino posible a la IP destino. Además, podemos saber
cuánto se tardó entre el envío de un paquete y la respuesta del router, que 
corresponde al RTT. Restando el RTT de un paquete del RTT del paquete con TTL
incrementado en uno podemos estimar el RTT entre un router y el siguiente
en la ruta.

Puede ocurrir que la IP destino no responda al paquete enviado, o que la IP
sea inalcanzable. Para salvarnos de estos casos, la herramienta incrementa
el TTL hasta 32, y si no logra alcanzar el destino hasta ese momento corta la
ejecución.

Para lidiar con variaciones en los RTT de las rutas y de los routers de cada hop
de cada paquete de TTL distinto,
se envían una cantidad configurable de paquetes por cada TTL, con valor por
defecto 32. Los valores de los RTT de estos paquetes serán luego procesados con
scripts en R para obtener un RTT entre routers aproximados.

El programa escribe la IP destino, el RTT y el TTL de cada paquete que envía
a un archivo CSV. Si el destino no es sabido debido a que el router que envió
el mensaje de \emph{Time Exceeded} no incluyó su IP, se imprime "NA" en su 
lugar. 
