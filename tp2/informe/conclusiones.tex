La primera conclusión a la que podemos llegar es la de que los servicios
de ubicación según IP no son del todo confiables. De todas formas, pudimos 
realizar teorías sobre las posiciones reales de los routers, y llegamos con
ellas a resultados de outliers un poco más plausibles. El algoritmo de
Cimbala demostró ser bastante exacto, si es que nuestras teorías fueron
correctas, excepto en el primer hop y en casos con muchas diferencias iguales
a 0, por el hecho de no tener suficientes muestras para detectar con seguridad
outliers. Además, la muestra de la Universidad de Pekín tuvo menos hops: 19 contra
los 23 y 28 de las demás. Podemos ver que cuantos más hops, más outliers se descubrían,
y eran outliers que se correspondían con las conexiones internacionales.
